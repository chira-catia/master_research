\label{subsec:ns2}

NS-2 was started as a version of the REAL network simulator. It is currently at its 
second iteration while its third(NS-3) is in active development and in a usable state.
Work on it started working in 1989 and it was quicly adopted by the research community.
Among the contributors to NS-2 have been Sun Microsystems, Xerox and Carnegie Mellon 
university. 

NS-2 simulations are written in a combination in of C/C++ and OTCL. C and C++ is 
used for definig protocols and libraries while OTCL is used to define and control
the simulation.

Basic support for mobile nodes in NS-2 is added through the MobileNode class. This extends
the generic Node class and adds support for mobilty of the node and the capability to
send and receive messages on a wireless medium. 
Example commands to set the position and the movement of a node:
\lstset{numbers=none,language=C,caption=Commands to set the position and movement of a node,label=lst:saddrule}
\begin{lstlisting}
	$node set X_ <x>
	$node set Y_ <y>
	$node set Z_ <z>
	$ns at $time $node setdest <x> <y> <speed>
\end{lstlisting}
As it can be seen, although we can specify the position of the node in three dimensions,
only two are used for the mobilty of nodes.

Support for simulating wireless sensor networks in NS-2 has been added by the 
MANNASIM project\footnote{\url{http://www.mannasim.dcc.ufmg.br/}}. It comes as
a patch for NS-2 which add support for wireless sensors. It includes a number of
new classes:
\begin{itemize}
	\item SensorNode class which is built on top of NS-2's MobileNode 
to include characteristics specific to wireless sensors such as power 
consumption, memory and cpu.
	\item BatteryClass, extenden from EnergyModel, which is used to simulate
the battery of a node
	\item DataGenerator, used for simulating the data with which the sensor is
working. Can be used to simulate programmed, continuous, on demant and event driven
data sources.
\end{itemize}

An important part of wireless sensor simulators is the environment in which the sensors
are placed. It can affect the power needed to transmit a packet, the placement of nodes
and the accurate determination of the position of neighbouring nodes. It is possible to
simulate some of these in NS-2 but not in a detailed way.
