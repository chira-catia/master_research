\label{subsec:jsim}
J-sim\footnote {original web site \url{http://j-sim.cs.uiuc.edu/}, moving to 
\url{http://sites.google.com/site/jsimofficial/}} is not a network simulator
by itself, but a component based development environment. Component based
development focuses on separation of concerns between the different subsystems
of a software system. In term of a wireless sensor node it means that only 
battery will be in charge of the flow of electrecity through the system.
This approach tries to mimic an integrated circuit. On top of the framework 
provided by J-Sim, a wireless network simulator has been built.

Following the component model, every object from a network is represented
by a component. For example a node is a component, also a routing protocol
is a component. The behaviour of such a component is specified by a contract,
somewhat similar to a IC specification. Bindind between components and contracts
is done in two phases. Contract bindind is done at design time while component
binding is done on system integration time. This means that a component can be
designed without interference from other components and that we can use any type
of component that fulfills a contract. For example we are not limited to a nickel
battery, but we could we lithium-ion one or even a lemon battery, because they all
implement the same contract.
