\label{subsec:tossim}
TOSSIM\cite{tossim} is a simulator for TinyOS sensor networks.
All the protocol and systems simulated using TOSSIM are written in nesC \cite{nesC},
an extension to C programming language designed to meet the specification and 
restrictions of TinyOS. At a first glance this might seem a disadvantage, but 
the code can then be reused on real-world motes running TinyOS
without further changes.
TOSSIM provides debugging facilities having several debugging modes 
(such as boot, clock,task,led,...) some of which are being used in TinyOS code
and others are reserved for applications components. Compiling TinyOS for mote
hardware removes the debug statements.
It also provides support for network monitoring and packet injection
through SerialForwarder\footnote{\url{http://docs.tinyos.net/index.php/Mote-PC_
serial_communication_and_SerialForwarder}}, the TinyOS interface tool.
//
For radio communication, TOSSIM provides two modes: the simple mode when bits
are transmitted without errors regardless of the distance between of the transmitting
motes, and the lossy mode, when for every pair of motes a number between 0 and 1 
denotes the probability with which enery transmitted bit will be corrupted. 

