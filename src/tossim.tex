\label{subsec:tossim}
TOSSIM\cite{tossim} is a simulator for TinyOS sensor networks.
All the protocols and systems simulated using TOSSIM are written in nesC \cite{nesC},
an extension to C programming language designed to meet the specification and 
restrictions of TinyOS. 

TOSSIM provides debugging facilities having several debugging modes 
(such as boot, clock,task,led,...) some of which are being used in TinyOS code, 
while others are reserved for applications components. Compiling TinyOS for mote
hardware removes the debug statements.
It also provides support for network monitoring and packet injection
through SerialForwarder\footnote{\url{http://docs.tinyos.net/index.php/Mote-PC_
serial_communication_and_SerialForwarder}}, the TinyOS interface tool.

For radio communication, TOSSIM provides two modes: the simple mode when bits
are transmitted without errors regardless of the distance between the transmitting
motes, and the lossy mode, when for every pair of communicating motes a number between 0 and 1 
denotes the probability with which every received bit will be corrupted. These 
mappings are defined in a configuration file which is given as an argument at boot
time. The values specified in this configuration file can be changed at runtime
by sending control messages through a TCP socket to TOSSIM.
 Except for this lossy mode, the degradation of signal can be modeled using
LossyBuilder which limits the transmitting area of every mote to a radius of 50
 feet and, at the same time, takes into consideration the rates defined as 
described above.
The values read from ADC are 10 bit values generated either random or generic, set
by default to random, but which can be controlled through control messages.

TOSSIM models the EEPROM memory with a memory-mapped file which, by default, is
unmapped at the end of simulation but it can be configured to be persistent if a
named file is specified at run-time.

TinyViz is an extensible GUI which, besides offering a visual layout of the 
WSN, offers support for debugging and interaction with TOSSIM. Also, traffic can
be seen using TinyViz and extra functionality added through plugins
\footnote{\url{http://www.tinyos.net/tinyos-1.x/doc/tutorial/lesson5.html}}.
A model of a TOSSIM network is defined by the following structure:
\lstset{numbers=none,captionpos=b,frame=single,language=C,caption=Structure for defining a network in TOSSIM,label=lst:tossimnet}
\begin{lstlisting}
typedef struct {
    void(*init)();
    void(*transmit)(int, char); //int moteID, char bit
    void(*stop_transmit)(int);  //int moteID
    char(*hears)(int);          //char bit, int moteID
    bool(*connected)(int,int);  //int moteID1, int moteID2
    link_t*(*neighbors)(int);   //int moteID
} rfm_model;
\end{lstlisting}

