\label{subsec:architecture}

Our simulator has a modular design. Each type
of component, a battery or a sensor, will a have a template containing a
very basic implementation of it. For example a transceiver template will
have a template consisting of the send() and recv() methods, but more
usefull implementations will be built on this to take into account the
environment and power consumption. On top of this template many different
implementations may be derived; for example a lithium-ion battery or a solar
powered one. Each component will provide an access interface for other components
(a sensor might have a read_data() interface) and they will not depend on other 
components. This way we will be able to simulate a wide range of platforms by
combining different components in different slots.

All the activity on one node will be coordinated by a CPU component. This will provide
an API to the routing protocol running on the node for activities such as allocating memory 
and sending and receiving packets. To simulate a slower node, the code running on the 
node will be ran using the \textit{ptrace} system call. The protocol will set in the beginning
breakpoints at every point in the program and the parent process(the node) will limit
its speed based on a configured value. We do not see any reasons to have more than one
type of CPU so only one will be available, but its performance and available memory
will be configurable.

Data through the network will not come only from the routing protocol, but sensors on the 
node will be able to generate realistic data that will accurately model the normal load
on such a network. This is important in order to see how protocols behave under load 
because some of them work by aggregating data.

It is important to be able to gather information from the network(number of packets sent
or received, battery life, power consumption per device) so there will be a central agent
responsible for gathering this information. Each component will send the statistics gathered
by them to the agent who will be able to present them in a more usefull form (bars and pies).
