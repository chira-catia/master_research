\label{subsec:architecture}

The simulator we will be building will have a modular design. Each type
of component, a battery or a sensor, will a have a template containing a
very basic implementation of it. For example a tranceiver template will
have a template consisting of the send() and recv() methods, but more
usefull implementations will build on this to take into account the
environment and power consumption. On top of this template many different
implementations may be derived; for example a lithium-ion battery or a solar
powered one. Each component will provide an access interface for other components
(an sensor might have a read_data() interface) and they will not depend on other 
components. This way we will be able to simulate a wide range of platforms by
combining different components in different slots.

All the activity on one node will be coordinated by a CPU component. This will provide
a API to routing protocol running on the node for activities such as allocating memory 
and sending and receiving packets. To simulate a slower node, the code running on the 
node will be ran using the ptrace system call. The protocol will set in the begining
breakpoints at every point in the program and the parent process(the node) will limit
its speed based on a configured value. We don't see any reasons to have more than one
type of CPU so only one will be available but have it performance and available memory
configurable.
